\documentclass[10pt, oneside]{article}   	
\usepackage[left=25mm,top=25mm,right=25mm,bottom=25mm]{geometry}   
\geometry{a4paper}
\usepackage{algorithm,algpseudocode}  
\usepackage{graphicx}					
\usepackage{amssymb}
\usepackage{amsmath}
\usepackage{gensymb}
\usepackage{multirow}
\usepackage{url}
\usepackage{titlesec}
\usepackage{subcaption}
\usepackage[export]{adjustbox}
\usepackage[parfill]{parskip}
\usepackage{cite}
\usepackage{hyperref}
\usepackage{array}
\usepackage{siunitx}
\usepackage{listings}
\usepackage{fixltx2e}
\usepackage{color}
\usepackage{diagbox}
\usepackage[usenames,dvipsnames,svgnames,table]{xcolor}
\usepackage[parfill]{parskip}
\setlength{\headsep}{5pt}
\graphicspath{ {images/} }
\lstset{
  basicstyle=\ttfamily,
  columns=fixed,
  fontadjust=true,
  basewidth=0.5em
}
\setcounter{topnumber}{3}
\setcounter{bottomnumber}{1}
\setcounter{totalnumber}{3}
\let\OLDthebibliography\thebibliography
\renewcommand\thebibliography[1]{
  \OLDthebibliography{#1}
  \setlength{\parskip}{0pt}
  \setlength{\itemsep}{1pt plus 0.5ex}
}
\renewcommand\thesection{\alph{section}}
\renewcommand\thesubsection{\thesection.\roman{subsection}}
\titleformat{\section}
{\normalfont\large\bfseries}{\thesection}{1em}{}
\titleformat{\subsection}
{\normalfont\normalsize\bfseries}{\thesubsection}{1em}{}
\title{\vspace{-1cm}Biomedical Information Processing (R214): main assignment report}
\author{Chongyang Shi \emph{(cs940)}}
\date{\today}							
\begin{document}
\maketitle

For the main course assignment, I am undertaking the second practical option (\textbf{1.2}): \emph{extracting chemical-disease associations from the biological literature}.

\section{Improving the Conditional Random Fields named entity recognizer}
\subsection{Ablating features from the original feature set}

Based on the default \emph{n}-gram feature set in the feature extraction script, the script was modified to ablate each feature in turn. To provide a better understanding of the effects by offsets of surface form words, all surface forms of words within an offset of 1 were knocked out from the templates first, then just the surface forms of words before and after the current word (-1/1). Other features including the lemma, phonetic coding ($soundex$), part-of-speech, and chunk in IOB2 notation of the current word only. The resulting precisions, recall rates, and $F_1$-scores from ablating each feature are presented separately in Figures \ref{fig:ablation1}, \ref{fig:ablation2}, and \ref{fig:ablation3}. For each named entity class as well as the average performance, with none-ablated as reference, improved performance due to ablation are presented in \textbf{bold}.

\begin{figure}[h]
\begin{center}
\fontsize{9}{11}\selectfont
\begin{tabular}{|*{8}{c|}}\hline
\backslashbox{Class}{Ablated} & None & $word$ (all) & $word$ (-1/1) & $lemma$ & $soundex$ & $pos$ & $chunk$ \\ \hline
B-Chemical & 0.9178 & \textbf{0.9345} & \textbf{0.9409} & 0.9056 & 0.9015 & \textbf{0.9495} & \textbf{0.9210} \\ \hline
O                 & 0.9560 & 0.9471 & 0.9498 & 0.9540 & 0.9531 & 0.9499 & 0.9557 \\ \hline
B-Disease   & 0.8403 & 0.8242 & 0.8223 & \textbf{0.8418} & 0.8387 & \textbf{0.8412} & 0.8396 \\ \hline
I-Disease    & 0.7404 & 0.7152 & 0.7167 & \textbf{0.7467} & \textbf{0.7506} & \textbf{0.7631} & \textbf{0.7509} \\ \hline
I-Chemical  & 0.7556 & 0.6488 & 0.6745 & \textbf{0.7569} & \textbf{0.7612} & \textbf{0.7906} & \textbf{0.7682} \\ \hline
\textbf{Macro-average} & 0.8420 & 0.8142 & 0.8209 & 0.8410 & 0.8410 & \textbf{0.8589} &\textbf{ 0.8471} \\ \hline
\end{tabular}
\caption{\label{fig:ablation1} Resulting \textbf{precisions} on different named entity classes from ablating individual features from the original feature set. }
\end{center}
\end{figure}

\begin{figure}[h]
\begin{center}
\fontsize{9}{11}\selectfont
\begin{tabular}{|*{8}{c|}}\hline
\backslashbox{Class}{Ablated} & None & $word$ (all) & $word$ (-1/1) & $lemma$ & $soundex$ & $pos$ & $chunk$ \\ \hline
B-Chemical & 0.6664 & 0.5583 & 0.5955 & 0.6564 & 0.6520 & 0.5702 & 0.6652 \\ \hline
O                 & 0.9888 & 0.9888 & \textbf{0.9889} & 0.9888 & 0.9887 & \textbf{0.9908} & \textbf{0.9894} \\ \hline
B-Disease   & 0.6011 & 0.5514 & 0.5672 & 0.5669 & 0.5561 & 0.5806 & 0.5992 \\ \hline
I-Disease    & 0.6018 & 0.5530 & 0.5607 & 0.5993 & 0.5952 & \textbf{0.6029} & 0.5952 \\ \hline
I-Chemical  & 0.5961 & 0.5114 & 0.5275 & 0.5950 & 0.5910 & 0.5938 & \textbf{0.5990} \\ \hline
\textbf{Macro-average} & 0.6908 & 0.6326 & 0.6479 & 0.6813 & 0.6766 & 0.6677 & 0.6896 \\ \hline
\end{tabular}
\caption{\label{fig:ablation2} Resulting \textbf{recall rates} on different named entity classes from ablating individual features from the original feature set. }
\end{center}
\end{figure}

\begin{figure}[h]
\begin{center}
\fontsize{9}{11}\selectfont
\begin{tabular}{|*{8}{c|}}\hline
\backslashbox{Class}{Ablated} & None & $word$ (all) & $word$ (-1/1)& $lemma$ & $soundex$ & $pos$ & $chunk$ \\ \hline
B-Chemical & 0.7721 & 0.6992 & 0.7294 & 0.7611 & 0.7567 & 0.7125 & \textbf{0.7725} \\ \hline
O                 & 0.9721 & 0.9675 & 0.9690 & 0.9711 & 0.9706 & 0.9699 & \textbf{0.9723} \\ \hline
B-Disease   & 0.7008 & 0.6607 & 0.6713 & 0.6776 & 0.6687 & 0.6870 & 0.6993 \\ \hline
I-Disease    & 0.6640 & 0.6238 & 0.6292 & \textbf{0.6649} & 0.6639 & \textbf{0.6736} & \textbf{0.6641} \\ \hline
I-Chemical  & 0.6665 & 0.5720 & 0.5920 & 0.6662 & 0.6654 & \textbf{0.6782} & \textbf{0.6731} \\ \hline
\textbf{Macro-average} & 0.7551 & 0.7046 & 0.7182 & 0.7451 & 0.7451 & 0.7443 & \textbf{0.7562} \\ \hline
\end{tabular}
\caption{\label{fig:ablation3} Resulting \textbf{$F_1$-scores} on different named entity classes from ablating individual features from the original feature set.}
\end{center}
\end{figure}

Surprisingly, for chemicals at the beginning of entities (B-Chemcal), the precision (correct tags among those tagged) increased substantially when the surface form word feature is ablated. Ablating only the surface forms of before and after tokens (-1/1) produced slightly higher precision than ablating that of before, current, and after tokens. This is however accompanied by substantially reduced precisions on all other named entity classes, as well as reduced recall rate (correct tags among all relevant inputs that can be tagged) nearly across the board. As B-chemicals already bear a fairly high precision (91.78\%), it is not advisable to ablate the surface forms in exchange for lowering recall into the 50\%s. 

Ablation of the lemma (base word) and the phonetic coding ($soundex$) yielded minimal improvements to precisions on some named entity groups but minimal reductions on others. Recall rates all reduced by very small margins. Based on a generally negative outlook on the $F_1$-scores (combined metric of precision and recall), it is advisable not to alate either of he two features.

Ablating the part-of-speech produced the greatest precision improvements to most groups, but mostly lowered the recall rate substantially. This is also reflected in the overall negative outlook on the combined $F_1$-scores. Therefore it is not advisable to ablate the part-of-speech feature. 

Finally, ablating the chunk information from the feature set improved the precision without significantly affecting the recall rate in most cases, resulting in improved $F_1$-scores for all named entity classes barring diseases within entities (I-Disease) with a minimal decrease. Therefore, it is advisable to ablate the chunk information from the feature set used.

Vertically, precision and recall rates of terms outside entities (O) are high and only very minimally affected by ablating any of the features, which is generally expected in entity recognition operations due to the abundance of outside tokens between short named entities \cite{ratinov2009design}.

\bibliographystyle{IEEEtran}
\footnotesize{\bibliography{report}}
\end{document}  
